\documentclass{article}

\usepackage[utf8]{inputenc}
\usepackage{lmodern}
\usepackage{amsmath}
\usepackage{graphicx}
\usepackage{url}

\usepackage{alltt}
\usepackage{verbatim}

\usepackage{listings}
\usepackage{color}
\usepackage[    colorlinks,%
                citecolor=black,%
                filecolor=black,%
                linkcolor=black,%
                urlcolor=black  ]{hyperref}
\usepackage{attachfile}

\usepackage{layout}
\usepackage[top=3cm, bottom=3cm, left=3cm, right=3cm]{geometry}

\usepackage{enumitem}
\usepackage{listings}
\usepackage{caption}
\usepackage{subcaption}


\usepackage{color}
\definecolor{gris}{rgb}{0.97,0.97,0.97}
\definecolor{mymauve}{rgb}{0.58,0,0.82}
\definecolor{mygreen}{rgb}{0,0.6,0}

\lstset{numbers=left, tabsize=4, backgroundcolor=\color{gris},
frame=single, breaklines=true,
keywordstyle=\color{blue},
commentstyle=\color{mygreen},    % comment style
numberstyle=\footnotesize\color{black}, % the style that is used for the line-numbers
stringstyle=\color{mymauve},     % string literal style
stringstyle=\ttfamily,
framexleftmargin=6mm, xleftmargin=6mm,
language=Java}

\makeatletter
\def\clap#1{\hbox to 0pt{\hss #1\hss}}%
\def\ligne#1{%
\hbox to \hsize{%
\vbox{\centering #1}}}%
\def\haut#1#2#3{%
\hbox to \hsize{%
\rlap{\vtop{\raggedright #1}}%
\hss
\clap{\vtop{\centering #2}}%
\hss
\llap{\vtop{\raggedleft #3}}}}%
\def\bas#1#2#3{%
\hbox to \hsize{%
\rlap{\vbox{\raggedright #1}}%
\hss
\clap{\vbox{\centering #2}}%
\hss
\llap{\vbox{\raggedleft #3}}}}%
\def\maketitle{%
\thispagestyle{empty}\vbox to \vsize{%
\haut{}{\@blurb}{}
\vfill
\vspace{1cm}
\begin{flushleft}
\huge \@title
\end{flushleft}
\par
\hrule height 4pt
\par
\begin{flushright}
\Large \@author
\par
\end{flushright}
\vspace{1cm}
\vfill
\vfill
\bas{}{\@location, on \@date}{}
}%
\cleardoublepage
}
\def\date#1{\def\@date{#1}}
\def\author#1{\def\@author{#1}}
\def\title#1{\def\@title{#1}}
\def\location#1{\def\@location{#1}}
\def\blurb#1{\def\@blurb{#1}}
\date{\today}
\makeatother

\lstset{tabsize=4,
        basicstyle=\scriptsize,
        %upquote=true,
        aboveskip={1.5\baselineskip},
        columns=fixed,
        showstringspaces=false,
        extendedchars=true,
        breaklines=true,
        prebreak = \raisebox{0ex}[0ex][0ex]{\ensuremath{\hookleftarrow}},
		frame=single,
		numbers=left,
		numberstyle=\tiny\color{black},
        showtabs=false,
        showspaces=false,
        showstringspaces=false,
        identifierstyle=\ttfamily,
        keywordstyle=\color[rgb]{0.5,0,0.35}\bfseries,
        commentstyle=\color[rgb]{0.25,0.5,0.35},
        stringstyle=\color[rgb]{0.6,0,0},
        morecomment=[s][\color[rgb]{0.25,0.35,0.75}]{/**}{*/},
	language=Python
}

%opening
\title{LINGI2346 - Distributed application design \\
Mission 2}
\author{Debroux Léonard\\Thibaut Knop} 
\date{Année académique 2013-2014}

	
% \title{Report on Assignment 1}
% \author{\begin{tabular}{l p{2cm}} \\
% 	Léonard \textsc{Debroux} \\
% 	Kevin \textsc{Jadin} 
%     \vspace{3mm} \\
% \end{tabular} \\
% }
% \location{Louvain-La-Neuve}
% \blurb{%
% Université Catholique de Louvain\\
% \'Ecole Polytechnique de Louvain\\
% \textbf{LINGI 2143 : Concurrent System}\\[1em]
% }% 

\newcommand{\phasor}[2] {
  #1\phase{#2\degree}
}

\begin{document}

\begin{titlepage}
    \begin{center}
        {\huge LINGI2346 - Distributed application design}\\
        \vspace{0.4cm}
        
        {\Large {Teacher : Marc Lobelle}}\\
        \vspace{0.6cm}
        
        {\Large \textit{Problem 1: Sockets}}\\
        \vspace{1.2cm}

        \texttt{}\\
        \vspace{0.2cm}

        \includegraphics[height=10cm]{pageGarde.png}\\
        \vspace{0.1cm}
        {\Large \textbf{Universit\'e Catholique de Louvain}}
        \vspace{0.3cm}

        \vspace{2cm}
        Group 24\\
        \vspace{0.3cm}
        Léonard Debroux\\
        Thibaut Knop\\
        \vspace{0.4cm}
        2013-2014\\
    \end{center}
\end{titlepage}

%%%%%%%%%%%%%%%%%%%%%%%%%%%%%%%%TITRE ET TABLE DES MATIERES%%%%%%%%%%%%%%%%%%
%\maketitle
%\tableofcontents
%\addcontentsline{toc}{section}{ }
\newpage

%Write a narrative describing how you went about writing and testing each program you write. The narrative should describe what choices you faced and which you made. We should get a pretty good sense of what your program looks like from reading the narrative. So far as testing goes, describe how you went about choosing your test cases, what they were, and their results.

\section{Discussion}
In this section, we'll discuss about the different choices we've made regarding the implementation of a transfer protocol.

\subsection{Protocol choice}
We chose to use TCP and not UDP for several reasons:
\begin{itemize}
  \item The transmission errors and reordering are managed by TCP
  \item A stream oriented connection better suits the transmission of files.
\end{itemize}
In the implementation, the choice is made upon the socket creation.

\subsection{Operating process of the server}
The mode that seemed to be the better is the concurrent mode. \\
We chose this option to allow several client to connect to the server and use the transfer protocol.\\
The server thus behaves as telnet and forks upon receiving a connection attempt.

\subsection{Allocation of the various functions to the server and to the client}
Since it is a client-server architecture, the client initiates the connexion and the server replies to the request of the clients. 
Actually, the architecture of the protocol is very similar to the one used in telnet protocol.
The different functions can be modeled as follow (see Fig.~\ref{architecture}):


\subsection{How to send and execute commands to/by the server}
In order to send the command and be able to tackle the problem of having to deal with unknown length, we chose to create a header that is sent each time a command is sent.
The header is a small structure that contains the type of the command and the length of the argument. If the command does not need an argument, that length is set to $0$.\\

You can find the different possible values for the type in \verb@header.h@ 

When a command enters client-side, the command is parsed to identify which one it is, and then, a header is sent accordingly. If the command is supposed to be followed by an argument, the argument is retrieved and sent to the server.\\

Server-side, upon the reception of a header, depending on the type of command, either it is executed and the result is sent back, or the server waits to receive an argument, and then, upon reception, executes the command. If the commands is such that the client is waiting for a response, the same mechanism as above is used : the server send a header to the client, containing a special type (either \verb@GET_SIZE@ or \verb@ERRNO_RET@). Then the client know how much bytes have to been read on the socket.\\

Note that a special type for the header is \verb@ERRNO_RET@ and is used by the server to return some information about a failure to the client.

\subsection{How to transfer textual data of unspecified length from the client to the server or from the server to the client}

As specified above, the use of a header allows the server (resp. the client) to read the exact right number of bytes specified in the header. More precisely, the server expects to receive headers, specifying the command to run. In the case of the \verb@PWD@ and \verb@LS@ commands, the server don't have to receive any other arguments from the clients. It executes directly the corresponding command and return the result to the client. In the other cases, the server read exactly the number of bytes specified in the header.\\

For the distant \verb@LS@ command, we had different way to return the entries from the current directory to the client : 
\begin{itemize}
\item Concatenate all the entries into 1 single string of a specified length, typically a multiple of MSS (see next subsection), and precede this sending by a header containing the length of the string that contains the entries. However, if the concatenation of all entries into one string have a length superior to the specified length, the server has to cut the string into 2 packets to be sent, which can cause more complexity with the headers.
\item Send back every entries on the socket to the client, until a last packet containing the char \verb@\n@ is read by the client (the char \verb@\n@ cannot be contained in files name). The last therefore knows that it doesn't have to wait for other entries anymore. Each entries is sent into a 256 bits initialized string, which allows the client to knows how much bytes it have to read at each time. The value of 256 bits actually corresponds to the maximum filename length, \verb@NAME_MAX@, defined in \verb@limits.h@.
\end{itemize}

We chose the second solution more because we hadn't thought to the first one! If we had a little more time, we would probably change for the first solution : it is probably best to avoid to send to much small packets when it is not required.

\subsection{How to transfer a binary file of length unknown a priori}
This topic refers directly to the usage of \verb@GET@ and \verb@PUT@. As the behaviour of the two commands is very similar as they basically perform the same action except from the direction of the information, we'll describe the behavior of \verb@PUT@ only. \\

The actions performed by the client are the following:
\begin{itemize}
    \item header sent to inform of the command and of the length of the filename;
    \item message sent to give the filename;
    \item file is opened in binary read mode;
    \item header sent to tell the number of full size packets to be sent;
    \item loop sending full size packets;
    \item header informing of the length of the last packet sent;
    \item message containing the last packet.\\
\end{itemize}

The actions performed by the server are the following:
\begin{itemize}
    \item recognize the command to execute;
    \item receive the filename;
    \item receive the number of full size packets to be received;
    \item create/open a file in binary write mode with the right filename;
    \item loop to receive the full size packets
    \item receive a header informing of the length of the last packet to receive
    \item receive the last packet of the file
\end{itemize}

The size for the packets is set at 1072 bytes as it corresponds to two time the maximum segment size that ipv4 host are required to be able to handle.

\subsection{Can one do something useful with the OOB data within the framework of this problem ?}

\section{Presentation}
\subsection{Protocols implementing the various functions}
\subsection{User guides of the client and server}
\subsection{The source code directory and instruction for building the executable files}
\begin{itemize}
\item \verb@header.h@:
\item \verb@utils.c@:
\item \verb@myftp.c@:
\item \verb@myftpd.c@:
\item \verb@Makefile@:

\end{itemize}
\subsection{Commented listings of the programs}
\end{document}


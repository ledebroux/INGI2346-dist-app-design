\documentclass{article}

\usepackage[utf8]{inputenc}
\usepackage{lmodern}
\usepackage{amsmath}
\usepackage{graphicx}
\usepackage{url}

\usepackage{alltt}
\usepackage{verbatim}

\usepackage{listings}
\usepackage{color}
\usepackage[    colorlinks,%
                citecolor=black,%
                filecolor=black,%
                linkcolor=black,%
                urlcolor=black  ]{hyperref}
\usepackage{attachfile}

\usepackage{layout}
\usepackage[top=3cm, bottom=3cm, left=3cm, right=3cm]{geometry}

\usepackage{enumitem}
\usepackage{listings}
\usepackage{caption}
\usepackage{subcaption}


\usepackage{color}
\definecolor{gris}{rgb}{0.97,0.97,0.97}
\definecolor{mymauve}{rgb}{0.58,0,0.82}
\definecolor{mygreen}{rgb}{0,0.6,0}

\lstset{numbers=left, tabsize=4, backgroundcolor=\color{gris},
frame=single, breaklines=true,
keywordstyle=\color{blue},
commentstyle=\color{mygreen},    % comment style
numberstyle=\footnotesize\color{black}, % the style that is used for the line-numbers
stringstyle=\color{mymauve},     % string literal style
stringstyle=\ttfamily,
framexleftmargin=6mm, xleftmargin=6mm,
language=Java}

\makeatletter
\def\clap#1{\hbox to 0pt{\hss #1\hss}}%
\def\ligne#1{%
\hbox to \hsize{%
\vbox{\centering #1}}}%
\def\haut#1#2#3{%
\hbox to \hsize{%
\rlap{\vtop{\raggedright #1}}%
\hss
\clap{\vtop{\centering #2}}%
\hss
\llap{\vtop{\raggedleft #3}}}}%
\def\bas#1#2#3{%
\hbox to \hsize{%
\rlap{\vbox{\raggedright #1}}%
\hss
\clap{\vbox{\centering #2}}%
\hss
\llap{\vbox{\raggedleft #3}}}}%
\def\maketitle{%
\thispagestyle{empty}\vbox to \vsize{%
\haut{}{\@blurb}{}
\vfill
\vspace{1cm}
\begin{flushleft}
\huge \@title
\end{flushleft}
\par
\hrule height 4pt
\par
\begin{flushright}
\Large \@author
\par
\end{flushright}
\vspace{1cm}
\vfill
\vfill
\bas{}{\@location, on \@date}{}
}%
\cleardoublepage
}
\def\date#1{\def\@date{#1}}
\def\author#1{\def\@author{#1}}
\def\title#1{\def\@title{#1}}
\def\location#1{\def\@location{#1}}
\def\blurb#1{\def\@blurb{#1}}
\date{\today}
\makeatother

\lstset{tabsize=4,
        basicstyle=\scriptsize,
        %upquote=true,
        aboveskip={1.5\baselineskip},
        columns=fixed,
        showstringspaces=false,
        extendedchars=true,
        breaklines=true,
        prebreak = \raisebox{0ex}[0ex][0ex]{\ensuremath{\hookleftarrow}},
		frame=single,
		numbers=left,
		numberstyle=\tiny\color{black},
        showtabs=false,
        showspaces=false,
        showstringspaces=false,
        identifierstyle=\ttfamily,
        keywordstyle=\color[rgb]{0.5,0,0.35}\bfseries,
        commentstyle=\color[rgb]{0.25,0.5,0.35},
        stringstyle=\color[rgb]{0.6,0,0},
        morecomment=[s][\color[rgb]{0.25,0.35,0.75}]{/**}{*/},
	language=Python
}

%opening
\title{LINGI2346 - Distributed application design \\
Mission 2}
\author{Debroux Léonard\\Thibaut Knop} 
\date{Année académique 2013-2014}

	
% \title{Report on Assignment 1}
% \author{\begin{tabular}{l p{2cm}} \\
% 	Léonard \textsc{Debroux} \\
% 	Kevin \textsc{Jadin} 
%     \vspace{3mm} \\
% \end{tabular} \\
% }
% \location{Louvain-La-Neuve}
% \blurb{%
% Université Catholique de Louvain\\
% \'Ecole Polytechnique de Louvain\\
% \textbf{LINGI 2143 : Concurrent System}\\[1em]
% }% 


\begin{document}

\begin{titlepage}
    \begin{center}
        {\huge LINGI2346 - Distributed application design}\\
        \vspace{0.4cm}
        
        {\Large {Teacher : Marc Lobelle}}\\
        \vspace{0.6cm}
        
        {\Large \textit{Problem 2: RPC}}\\
        \vspace{1.2cm}

        \texttt{}\\
        \vspace{0.2cm}

        \includegraphics[height=10cm]{pageGarde.png}\\
        \vspace{0.1cm}
        {\Large \textbf{Universit\'e Catholique de Louvain}}
        \vspace{0.3cm}

        \vspace{2cm}
        Group 24\\
        \vspace{0.3cm}
        Léonard Debroux\\
        Thibaut Knop\\
        \vspace{0.4cm}
        2013-2014\\
    \end{center}
\end{titlepage}

%%%%%%%%%%%%%%%%%%%%%%%%%%%%%%%%TITRE ET TABLE DES MATIERES%%%%%%%%%%%%%%%%%%
%\maketitle
%\tableofcontents
%\addcontentsline{toc}{section}{ }
\newpage

%Write a narrative describing how you went about writing and testing each program you write. The narrative should describe what choices you faced and which you made. We should get a pretty good sense of what your program looks like from reading the narrative. So far as testing goes, describe how you went about choosing your test cases, what they were, and their results.

\section{Discussion}

\subsection{Allocation of the various functions to the server and to the client}
Thanks to the RPC model, we can call procedure outside of the application address space. For that purpose, we have to specify the data structure used as argument or return value of the remote procedures in the RPCL file (\verb@rpspec.x@), the \verb@main@ method of the client, as well as the remote procedures executed by the server side. \verb@RPCGEN@ help us by generating the code to handle the connections and exchange of data between the client and the remote process.\\

There are 2 types of actions possible for the client (\verb@myftp.c@) :
\begin{enumerate}
\item Perform a local command : the main directly utilizes the auxiliary functions that perform the requests and described in \verb@utils.c@
\item Perform a remote command on the server : the client call the remote procedures defined on the server side (\verb@myftpd.c@) thanks to the RPC model. The server side will call the auxiliary functions that perform the requests, as described in \verb@utils.c@.
\end{enumerate}

\subsection{Organisation of the transfer between client and server of data of unspecified length as parameters and results of remote procedure calls}
Concerning the reading of directory, for the \verb@ls@ remote procedure call, we choose to return a string containing all the content of the repertory.\\

Concerning the files however, it is a bit more complicated : when a client wants to retrieve a file, he remotely calls for the \verb@rget_1@ procedure with a \verb@file_desc@ structure as argument. This structure contains the filename as well as an offset. The procedure reads \verb@PSIZE@ bytes (specified in the server code) from the specified file from the specified offset. It returns a \verb@file_part@ structure containing 
\begin{itemize}
\item An \verb@int@ indicating if it is the last chunk of the file or not : more precisely, this \verb@int@ contains always 0, unless there is an error or this is the last chunk of the file. In the former case, then the \verb@int@ contains the length of the error string, and in the former, it contains the length of the chunk. This allows the client to know that it is the last chunk, even the length of this chunk is, by chance, the same length as the previous chunk.
\item The content of the chunk it reads, with an opaque type. The opaque type is used in RPCL to describe untyped data, that is, sequences of arbitrary bytes. We couldn't choose the string type since a file can contains a 0-valued byte, and this byte in a string is considered as the end of the string. This chunk is a structure itself, containing the length of the read data as well as the data.
\end{itemize}

When receiving the answer, the client thus know if it was the last chunk or not. If not, he calls again for the 
 \verb@rget_1@ remote procedure but with the offset correctly set, and so on, until he receives the last chunk.\\
 
Note that the size of packets used by the server and the client can be different, since the length are included in the data structures returned by the remote procedures.\\

To choose the correct size of chunks, we thought that to big chunks will require to much memory as well as prevent multiple users to simultaneously retrieve a file : a second but later user should wait for the first transfer to be completed. Bigger the chunk is, more the later client should wait. By choosing for chunk of 1024 bytes, we ensure a certain fairness between the users. On the opposite, choosing to small packets could overflow the network quickly, which is not valuable.
 
\subsection{How to support simultaneous clients ?}
At first view, we thought we had different options to support simultaneous clients :
\begin{enumerate}
\item Keep a dictionary on the server side containing the state (i.e the current directory) for each client. At each remote procedure call from a client, just look-up in the dictionary to retrieve the state of the connection. However, this solution was totally not possible due the architecture of the RPC model : indeed, the server don't maintain any information about the clients remotely calling procedures. Actually, it is the job of the \verb@PortMapper@ locate the remote procedures on a particular host.
\item Another solution is to send the current state at each remote procedure call. For that purpose, we modified our data structures in the RPCL file by adding a \verb@string pwd<>@ field in almost each structure. Then for all remote procedure call, the client is sending its own server environment (i.e the current directory the server should have if the client was the only one). However, one question remains : how to initialize a new state when a new client call for a remote procedure? To address this issue, upon the first client connection, the initial directory of the server is stored in a variable, and each new client will get that value upon their connection so that each client begins with the same working directory.
\end{enumerate}
By examining the code, one can observe that we are calling \verb@PWD@ remote procedures, with the actually current directory in the argument. This is because we hadn't decided yet how to support simultaneous clients. If the solution we choose is the right one, it should be obvious that remotely calling for getting the current directory while we already have it is totally useless.

\section{Presentation}
\subsection{Communication protocol between the client and the server}

\subsection{User guides of the client and server}
\begin{description}
\item[Server] The only command you can perform for the server is to run it by launching \verb@./myftpd@. All the commands are sent remotely by the client.
\item[Client] As specified in the functional description of the application, the client is launched by running \verb@./myftp@ in the right corresponding directory (see next Subsection). The commands available are :
\begin{itemize}
\item \verb@pwd@: Display the current directory (of the environment) of the server
\item \verb@lpwd@: Display the current directory (of the environment) of the client
\item \verb@cd dir@: Change the current directory of the server into \verb@dir@. Not that the usage of relative and absolute paths, as well as \verb@~@ paths  are accepted.
\item \verb@lcd dir@: Change the current directory of the client into \verb@dir@. Not that the usage of relative and absolute paths, as well as \verb@~@ paths  are accepted.
\item \verb@ls@: Display the contents of the current directory of the server.
\item \verb@lls@: Display the contents of the current directory of the client.
\item \verb@get file@: The file \verb@file@ is copied from the current directory of the server towards the current directory of the client.
\item \verb@put file@: The file \verb@file@ is copied from the current directory of the client towards the current directory of the server.
\item \verb@bye@: Close the file transfer session
\end{itemize}
\end{description}
\subsection{The source code directory and instruction for building the executable files}
The directory containing the source codes is named \verb@Mission3-Group24@.  \\
It contains the following files : 
\begin{itemize}
\item \verb@utils.c@: contains a set of methods which are useful for both client and server. All the methods are designed to be used and called either by the client and the server. 
\item \verb@myftp.c@: contains the code for the client.
\item \verb@myftpd.c@: contains the code for the server.
\item \verb@Makefile@: automate the compilation and the linkage of the .c sources to produce executable files, \verb@myftp@ and \verb@myftpd@ respectively, as well as the generation of the \verb@XDR@ procedures, the server and client stub.
\item \verb@rpspec.x@ : specifies the names, program, version and procedure numbers of the remote procedures.
\item \verb@rpspec_clnt.c@ : contains the generated code for the client stub, that contains the real remote procedure calls for the client process and looks like a local implementation of the remote procedure.
\item \verb@rpspec_svc.c@ : contains the generated code for the server stub.
\item \verb@rpspec.h@ : a common header file used by both client and server stub.
\end{itemize}
In order to build the executable files, all is needed is to run the \verb@Makefile@ by running the \verb@make@ command in the current directory (\verb@Mission3-Group24@).\\

To run the client and the server, just run \verb@./myftp@ and \verb@./myftpd@ respectively, by starting the server first.
\subsection{Commented listings of the programs}
 \subsubsection{.x file}~
   \lstinputlisting{../myftp/rpspec.x}
 
  \subsubsection{RPCGEN generated files}~
  	      \lstinputlisting{../myftp/rpspec.h}
 	     \lstinputlisting{../myftp/rpspec_clnt.c}
	     \lstinputlisting{../myftp/rpspec_svc.c}
 
 \subsubsection{Client specific code}~
    \lstinputlisting{../myftp/myftp.c}
    \subsubsection{Server specific code}~
    \lstinputlisting{../myftp/myftpd.c}
    \subsubsection{../Other}~
    \lstinputlisting{../myftp/utils.c}
    \lstinputlisting{../myftp/utils.h}
     \lstinputlisting{../myftp/Makefile}
\end{document}

